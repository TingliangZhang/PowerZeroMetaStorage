% !TEX TS-program = xelatex
% !TEX encoding = UTF-8

% This is a simple template for a XeLaTeX document using the "ctexart" class by 张庭梁,
% with the fontspec package to easily select fonts.

\documentclass{ctexart} % 使用ctex 适配过的article 文档类
%\usepackage{fontspec} % Font selection for XeLaTeX; see fontspec.pdf for documentation
%\defaultfontfeatures{Mapping=tex-text} % to support TeX conventions like ``---''
%\usepackage{xunicode} % Unicode support for LaTeX character names (accents, European chars, etc)
%\usepackage{xltxtra} % Extra customizations for XeLaTeX
% other LaTeX packages.....
\usepackage{geometry} % See geometry.pdf to learn the layout options. There are lots.
\geometry{a4paper} % or letterpaper (US) or a5paper or....
%\usepackage[parfill]{parskip} % Activate to begin paragraphs with an empty line rather than an indent

\usepackage{graphicx} % support the \includegraphics command and options
\graphicspath {{fig/}} % 设置图片目录
\usepackage{cite}  %某一处需要同时引用多篇文献,在正文中直接引用   \cite{1}, \cite{2}   会生成这种样式:[1][2]   如果是3篇及以上    \cite{1,2,3}   会生成这种样式: [1-3]

\usepackage[cache=false]{minted}  %You must invoke LaTeX with the -shell-escape flag.
%如果出现一堆 Undefined control sequence.  那应该是没加 [cache=false]
%Package minted Error: You must have `pygmentize' installed to use this packag
%Solution:应该是Python的包Pygments的问题了,用pip安装了这个包
%   pip install pygments


\title{研究专题一:稀疏技术研究}
\author{张庭梁}
%\date{} % Activate to display a given date or no date (if empty),
         % otherwise the current date is printed 

\begin{document}
\maketitle

摘要:找到一套方法,用C语言编程实现大电网Y的存储和检索,实现排零存储和排零运算。设计算例,在研究报告中,阐述研究目的、方法、程序框图、算例结果和分析结论,并注明参考文献。


关键词:\LaTeX

%\tableofcontents % 这里是目录


\section{节点导纳矩阵定义及性质}
节点导纳矩阵$Y_{\mathrm{n}}$定义如下

\begin{equation}
Y_{n} \dot{U}_{n}=\dot{I}_{n}
\end{equation}

其中:

$Y_{\mathrm{ii}}$:节点i的自导纳,i所接所有支路导纳之和(含:串、并联)

$Y_{\mathrm{ij}}$:节点i、j的互导纳,i、j支路导纳负值。若i、j间没有直联支路:=0 。在实际大电网中,很多$Y_{\mathrm{ij}}$是0,所以有必要实现排零储存和运算。

对于一个三节点系统(不包括参考节点)的$Y_{\mathrm{n}}$,有:

\begin{equation}
\left[ \begin{array}{l}{\mathrm{I}_{1}} \\ {\mathrm{I}_{2}} \\ {\mathrm{I}_{3}}\end{array}\right]=\left[ \begin{array}{lll}{\mathrm{Y}_{11}} & {\mathrm{Y}_{12}} & {\mathrm{Y}_{13}} \\ {\mathrm{Y}_{21}} & {Y_{22}} & {\mathrm{Y}_{23}} \\ {\mathrm{Y}_{31}} & {\mathrm{Y}_{32}} & {\mathrm{Y}_{33}}\end{array}\right] \left[ \begin{array}{l}{\mathrm{U}_{1}} \\ {\dot{\mathrm{U}}_{2}} \\ {\dot{\mathrm{U}}_{3}}\end{array}\right]
\end{equation}

进一步$Y_{\mathrm{n}}$为:

\begin{equation}
\left[ \begin{array}{ccc}{\mathrm{y}_{10}+\mathrm{y}_{12}+\mathrm{y}_{13}} & {-\mathrm{y}_{12}} & {-\mathrm{y}_{13}} \\ {-\mathrm{y}_{12}} & {\mathrm{y}_{20}+\mathrm{y}_{12}+\mathrm{y}_{23}} & {-\mathrm{y}_{23}} \\ {\mathrm{-y}_{13}} & {-\mathrm{y}_{23}} & {\mathrm{y}_{30}+\mathrm{y}_{13}+\mathrm{y}_{23}}\end{array}\right]
\end{equation}

支路对$Y_{\mathrm{n}}$的贡献:

\begin{equation}
Y_{n}=\sum_{l=1}^{b} \left[ \begin{array}{cc}{y_{l}} & {-y_{l}} \\ {-y_{l}} & {y_{l}}\end{array}\right]
\end{equation}

总结:$Y_{\mathrm{n}}$性质
\begin{enumerate}
\item n×n阶对称复数方阵
\item 稀疏,很多零元。
\item 有接地支路时非奇异,没有接地支路时奇异。
\item 对称性很好,可以只储存上三角矩阵。
\end{enumerate}

\section{内存块置0的三种方法}

C/C++中内存块置0的三种方法:memset,ZeroMemory,SecurZeroMemory\cite{CN}。

一般情况,如初始化内存块的时候,用ZeroMemory。销毁内存块中储敏感数据时或者释放存有敏感数据(比如密码,密钥等)的内存块前应使用使用SecurZeroMemory。如无特殊原因不使用“={ 0 }”。

使用memset函数将内存块置0是完全没有问题。memset的好处是跨平台比较容易,可是C/C++跨平台就是梦魇。在使用memset的时候有个小地方需要注意,W.Richard Stevens在《UNIX网络编程》中提到void *memset(void *dest, int c, size t count)的后两个参数容易写反,而且在编译时无法发现。

ZeroMemory宏,在底层就是由memset实现的。只是ZeroMemory易读性更好,更加健壮。或者说看起来更cool、更professional。在微软平台下的程序,推荐使用ZeroMemory。

SecurZeroMemory函数,可以看作是在安全方面加强版的ZeroMemory。细心的读者是否注意到ZeroMemory是宏,而SecurZeroMemory是函数?ZeroMemory在一定的编译优化条件下,使用ZeroMemory置0以后的内存块如果再也不被引用,ZeroMemory有可能会被“优化”掉而不执行。如果这块内存里存储的是用户的密码、加解密算法的密钥等敏感信息,就存在被黑客偷窥的可能。而SecurZeroMemory在任何条件下都不会被“优化”掉,所以在销毁内存块中储敏感数据时或者释放存有敏感数据的内存块前应使用SecurZeroMemory,而不是ZeroMemory。

至于"={ 0 }"的形式,尽量不要使用,不够直观。而且在内存对齐方面也存在一定问题。参考Raymond Chen的《Why do Microsoft code samples tend to use ZeroMemory instead of { 0 }?》


\section{节点导纳矩阵元素存贮的方法}

在大型电力系统节点导纳矩阵的形成及读写
过程中,如不考虑节点导纳矩阵Y 元素的稀疏性和
对称性,会导致大量零元素和对称元素的存贮以及
对称元素的计算,从而造成形成Y 阵所需时间较长、
存贮空间较大以及相应Y 阵数据文件的读写时间较
长。考虑Y 阵元素稀疏性及对称性的存贮方式不但
可大幅节省存贮单元,还可大大减少对Y 阵数据文
件的读写时间。

传统的不考虑元素稀疏性的Y 阵数组形式为
Y( n,2 n) ,存贮Y 阵的全部元素数值,元素的行、列
下标直接由元素在Y 阵中的位置确定。这种数组形
式简单直观,方便对Y 阵数据的处理,但由于大量零
元素以及对称元素的存贮而占据极大的存贮空间,
同时大大影响其相应数据文件的读写速度。

电力系统计算中考虑稀疏矩阵技术 的应用
广泛,存贮方法有坐标存贮法、顺序存贮法、链表
存贮法等。虽然这些存贮方法可节省不少
存贮单元,但存贮结构复杂,也未利用Y 阵的结构特
点及其元素的对称性,且对角元素与非对角元素、行
号与列号的分开存贮使得元素存取过程繁琐,不方
便对Y 阵数据的后续处理。

为分析比较,以IEEE - 118 系统的复数Y 阵元
素存贮为例,假设每个节点平均与6 条支路相连,各
种存贮方法所需的存贮单元总数如下。

1) 不考虑稀疏性时的传统存贮法。

Y 阵元素数组形式为Y( n,2n) ,1个数组可存贮Y 阵的全部元素,元素的行、列下标直接由元素在Y阵中位置确定,存贮单元总数S1为:

S1 = n × 2 × n = 118 × 2 × 118 = 27 848,其中n 为节点数。

2) 考虑稀疏性的坐标存贮法。

该方法分别需1 个对角元和1 个非对角元数组,存贮单元总数S2为:
S2 = ( n + 3N) = 118 × 2 + 3 × 118 × 6 × 2 =4484,其中n 为节点数或对角元数,N 为非零的非对角元数。
坐标存贮法的存贮单元数约占传统存贮法的16.10\%。该方法简单、直观、便于检索,但不便于计算。

3) 考虑稀疏性时的顺序存贮法。

该方法需1 个对角元和2 个非对角元数组,存贮单元总数S3为:

S3 = ( 2n + 2N) = 2 × 118 × 2 + 2 × 118 × 6 × 2 = 3 304。

顺序存贮法的存贮单元数约占传统存贮法的11.86\%。该方法与坐标存贮法相比,不够简单、直观,不便于检索,但便于计算,是当前采用较多的存贮法之一。

4) 考虑稀疏性时的链表存贮法。

该存贮方法也需1 个对角元和2 个非对角元数组,存贮单元总数S4为:

S4 = ( 2n + 2N + N) = ( 2n + 3N) = 2 × 118 × 2 +3 × 118 × 6 × 2 = 4 720

链表存贮法的存贮单元数约占传统存贮法的16. 95\%。这种存贮方法与顺序存贮法相似,结构略复杂,便于计算,应用广泛。

上述分析比较可以看出,传统的Y( n,2 n) 的存
贮方法最为简单直观、实用,但大量零元素的存贮使
得其所需的存贮单元数非常大。考虑稀疏性的坐标、
顺序或链表存贮法中由于对角元与非对角元、行号
与列号分开存贮,未利用Y 阵元素结构特点,虽可节
省大量的存贮单元,但这些存贮法结构复杂,不利于
数据的检索和利用,仍需较多存贮单元。



如图~\ref{1}所示
% 如果想不浮动使用[H!]命令
\begin{figure}[htbp]
\small
\centering
\includegraphics[width=8cm]{1.jpg}
\caption{嘿嘿嘿} 
\label{1}
\end{figure}




\section{算法实现}
\section{程序源码}


%也可以直接将代码\include{}替换
%如果本身就是被include的 就使用  \input{data/TGAMcode}
%因为include 不能嵌套
\begin{minted}{c++}
int main() {
    printf("hello, world");
    return 0;
}
\end{minted}


%\section{结论}
%\section{致谢}
%\section{附录}

%% 参考文献
% 注意:至少需要引用一篇参考文献,否则下面两行可能引起编译错误。
% 如果不需要参考文献,请将下面两行删除或注释掉。
% 数字式引用
%要在文中汉字之后 \cite{论文名}  引用才会出现在参考文献中
%可以到Google scholar上找bibtex格式引用
\bibliographystyle{plain}
\bibliography{ref/refs}

\end{document}
